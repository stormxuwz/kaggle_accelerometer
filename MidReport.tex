\documentclass{article}
\author{Wenzhao Xu}
\title{STAT 542 Mid Report}
\usepackage{amsmath}
\usepackage{graphicx}
\usepackage{subfig}
\usepackage{multirow}
\usepackage[top=1in, bottom=1in, left=1.25in, right=1.25in]{geometry}


\begin{document}
	\maketitle
	
	\section{Introduction} % (fold)
	\label{sec:introduction}
	\paragraph{} The data are about accelerometer data from mobile devices. The train data contain X,Y,Z axis acceleration values from 387 devices and testing data set consists of 90024 testing sequences. Each test sequence comes with a proposed device Id. The goal is to judge whether the claimed device is the true device that produces the test sequence. The main difficulties in this project are: (1) Data has noise and gaps (i.e. the sampling time interval is not constant); (2) Feature extraction; (3) Treat it as a multi-label classification problem or as a 0-1 classification problem by some label creating method. 
	
	% section introduction (end)
	
	\section{Method} % (fold)
	\label{sec:method}
	\paragraph{} The basic assumption is that each user will behavior similar if he/she is doing the same activity and he/she woule probabily do the same activity at the same time of day. 
	\paragraph{}First we treat it as a multi-label problems, that is trying to predict which device generates the test sequence. Given the training data of a certain device, we first divided the whole training data into pieces, each has 400 points (400 is a tuning parameters). For most devices, 400 sampling points will last for about 1 minute and we assume the user is doing the same activities in this 1 minute. Each piece represent an activity of the user, no matter what the activity is. So for train data of a device, we have several pieces and all of them have the same label as that device. Features are extracted from these pieces and also from the test sequences. In this way, we have training sequences with labels and test sequences. Certain classifier is trained and predicted the labels of test sequences.
	\paragraph{} We also tried to change the problem as a 0-1 classification problem. XXXXX, currently, we are still working on it. 
	
		\subsection{Data Preparation} % (fold)
		\label{sub:idea}
		\paragraph{} The data is imported into R by "ff" packages (SQL database is also available). Then the data is preprocessed through splitting, filling and smoothing to get the final training data and test data. 
		 \subsubsection{Splitting}
		\paragraph{}It is found that the training data of a given device don't have a constant sampling frequencies. This might be due to users' manually switching off device or the way Android system samples. Android OS would acquire data on certain events. So the whole training data might have large gaps. In order to deal with this, we calcuate the sampling time intervals. If the time interval is larger that 2 minutes, we split the training data into two data sets at this point. For each pieces, we further divide it into smaller sequences, each has 400 raw data points. At the end of this step, the training data from 387 devices are splited into almost 70000 sequences. 
		In addition, for test, similar splitting is used. If the time interval in one test sequence is larege than 3 minutes, we divide the test sequence into 2 parts and discard the parts with less data points.
		\subsubsection{Resampling}
		\paragraph{} In splitting step, we split the train data into sequences. However, in each sequences, the sampling frequencies are not the same, which is very hard for further analysis, especially for frequency analysis. So we do a resample step. A constant sampling time is determined based on the initial sampling time and the median value of original sampling intervals. Here median value is used to eliminate the influence of sampling interval outliers. Then, on the determined sampling time, the resampled acceralation value is calculated by linear interpolation based on two nearest sampling points. 
		\subsubsection{Smoothing}
		\paragraph{} Train data have noises such as large spikes. So a 5-point moving average algorithm is used to smooth the data. 5-point moving average is also used in other literatures. Figure X shows the data piece before and after resampling and smoothing steps. 
		% subsection idea (end)
	
		\subsection{Feature Extraction} % (fold)
		\label{sub:feature_extraction}
		
		\paragraph{}First, the total acceralation value is calculated by $A=\sqrt{(a_x^2+a_y^2+a_z^2)}$ and added. Based on literature research, 4 kind of features are extracted from the raw data. The first is mean and variance. The second is the correlation. The third is the frequency pattern and the last is the time. Totally, the number of features are 20, which is shown in Table 1. 
		\begin{table}
			\centering
			\begin{tabular}{p{3cm}|p{6cm}|p{4cm}}
			Kind (num)& Description & Physical Meanings \\ \hline
			Mean and Variance (8 features) & Mean and variance of acceleration values in each axis as well as the total accerlation $A$ & The habbit of how user put their cellphones and the strength of their activities \\ \hline
			Correlation (3 features) & the correlation coefficients between x and y, x and z, and y and z axis. & The users features when doing activities \\ \hline
			Frequency Features (8 features) & The mean value of the first 5 dominate frequencies (frequencies with highest amplitude) and mean value of energy in these frequencies & Users walking features.\\ \hline
			Time (1 feature) & mean time of day & User's habbit when doing such activity.
			\end{tabular}
		\end{table}

		% subsection feature_extraction (end)
	
		\subsection{Classifier} % (fold)
		\label{sub:classifier}
		\paragraph{} For simplicity, currently we just use KNN methods to detemine which device generate the test sequence. This is a multi-label classification problem, so other available methods include random forest and regression. 
		
		\paragraph{} However, if we treat the problem as a bi-label problem, other method such as SVM is also available. 
		% subsection classifier (end)
	% section method (end)
	
	\section{Result} % (fold)
	\label{sec:result}
	\paragraph{} The data preparation and features extraction from train data took about 20 mins and XX mins for test data. For simplicity, KNN is first used to predict the test sequences. The result is uploaded into the kaggle website to see the result. 
	
	
	% section result (end)
	
	\section{Future Work} % (fold)
	\label{sec:future_work}
	
	% section future_work (end)
	
\end{document}
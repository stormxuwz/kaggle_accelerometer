\documentclass{article}
\author{Wenzhao Xu}
\title{STAT 542 Mid Report}
\usepackage{amsmath}
\usepackage{graphicx}
\usepackage{subfig}
\usepackage{multirow}
\usepackage[top=1in, bottom=1in, left=1.25in, right=1.25in]{geometry}


\begin{document}
	\maketitle
	
	\section{Introduction} % (fold)
	\label{sec:introduction}
	\paragraph{} The data is going to 
	
	% section introduction (end)
	
	\section{Method} % (fold)
	\label{sec:method}
	\paragraph{} The basic assumption is that each user will behavior similar if he/she is doing the same activity and he/she probaby do the same activity at the same time of day. Given the training data of a certain device, we first divided the whole data into pieces, each has 300 points (300 is a tuning parameters). For most devices, 300 sampling points will last for about 1 minute and we assume the user is doing the same activities in this 1 minute. Each piece represent an activity of the user, no matter what the activity is. So for train data of a device, we have several pieces and all have the same label as that device. Features are extracted from these pieces and also from the test sequences. Then we can do the classification. 
	
		\subsection{Data Preparation} % (fold)
		\label{sub:idea}
		\paragraph{} The data is imported into R by "ff" packages (SQL database is also capcable). 
		
		% subsection idea (end)
	
		\subsection{Feature Extraction} % (fold)
		\label{sub:feature_extraction}
		
		\paragraph{}First, the total acceralation value is calculated by $A=\sqrt{(a_x^2+a_y^2+a_z^2)}$ and added. Based on literature research, 3 kind of features are extracted from the raw data. The first is mean and variance. The second is the correlation, the last is the frequency pattern. 
		\begin{table}
			\centering
			\begin{tabular}{c|p{5cm}|p{5cm}}
			Kind & Description & Physical Meanings \\ \hline
			Mean and Variance & Mean and variance of acceleration values in each axis as well as the total accerlation $A$ & The habbit of how user put their cellphones \\ \hline
			Correlation & the correlation coefficients between x and y, x and z, and y and z axis. & The users features when walking \\ \hline
			Frequency Features & The mean value of the first 5 dominate frequencies (frequencies with highest amplitude) and mean value of energy by these frequencies & Users walking features.
			\end{tabular}
			
		\end{table}
		
		
		
		% subsection feature_extraction (end)
	
		\subsection{Classifier} % (fold)
		\label{sub:classifier}
		
		% subsection classifier (end)
	
	% section method (end)
	
	\section{Result} % (fold)
	\label{sec:result}
	
	% section result (end)
	
	\section{Future Work} % (fold)
	\label{sec:future_work}
	
	% section future_work (end)
	
\end{document}